\documentclass[10pt, a4]{article}

\usepackage[T1]{fontenc}
\usepackage{cogsci}
\usepackage{pslatex}

\usepackage[round]{natbib}
\usepackage{graphicx}

\usepackage[english]{babel}

\usepackage{blindtext}

\graphicspath{{img/}}

\title{Do adults behave like children when under pressure?}

\author{{\large \bf Samuel Giacomelli (S.Giacomelli@student.rug.nl)} \\
University of Groningen}

\begin{document}

\maketitle

\begin{abstract}
    Do adults behave like children when under pressure? If children perform
    differently because they have more limited cognitive resources, like working
    memory, if you load adults working memory can you then get them to show childlike
    behavior/interpretations?   
\end{abstract}

\section{Introduction}
Symmetrical response (SR) and Logical reading (LG) in children and adults do to low level of \textit{COGNITIVE CONTROL}.
Multiple numbers (>= 6) of extra objects almost perfect LG (\cite{sugisaki2001quantification}).\\
Desired number of extra object should be the one that is capable to elicit both LG and SR responses.\\
Ability to flexibly switch perspective, children are inflexible (\cite{piaget1954language}).\\


\section{Method}

\subsection{Participants}
\blindtext

\subsection{Procedure}
\blindtext

\subsection{Materials}
Use both extra object and extra subject picture.\\
Place them in different places of the screen, not always in the bottom right corner. 
Three word recall task (\cite{cullum1993three})

\blindtext

\section{Results}
\blindtext


\section{Discussion}
\blindtext


\bibliographystyle{plainnat}

\setlength{\bibhang}{.125in}
\setlength{\bibindent}{-\bibhang}

\vfill
\pagebreak

\bibliography{giacomelli_samuel_pp}

\end{document}
