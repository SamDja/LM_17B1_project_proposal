\documentclass[10pt, a4]{article}

\usepackage[T1]{fontenc}
\usepackage{cogsci}
\usepackage{pslatex}
\usepackage{gb4e}
\noautomath

\usepackage[round]{natbib}
\usepackage{graphicx}

\usepackage[english]{babel}

\usepackage{blindtext}

\graphicspath{{img/}}

\title{Do adults behave like children when under pressure?}

\author{{\large \bf Samuel Giacomelli (S.Giacomelli@student.rug.nl)} \\
University of Groningen}

\begin{document}

\maketitle

\begin{abstract}
    Do adults behave like children when under pressure? If children perform
    differently because they have more limited cognitive resources, like working
    memory, if you load adults working memory can you then get them to show childlike
    behavior/interpretations?   
\end{abstract}

\section{Introduction}
Investigate on adult tendency to commit universal quantification when their cognitive resources are limited.\\
\textit{The present study aims at examining this hypothesis}.\\
\\
Symmetrical response (SR) and Logical reading (LG) in children and adults do to low level of \textit{COGNITIVE CONTROL}.
Multiple numbers (>= 6) of extra objects almost perfect LG (\cite{sugisaki2001quantification}).\\
Desired number of extra object should be the one that is capable to elicit both LG and SR responses.\\
Ability to flexibly switch perspective, children are inflexible (\cite{piaget1954language}).\\
Children between 4 and 5 yo as in (\cite{minai2012hinders}). Because they're able to provide both SR and LG responses\\
One group of adults 20-30.\\
\textit{Question-answer Requirement} -> sentences are interpreted as answers to particular questions. Maybe
under pressure the participant tend to find easier question, leading to commit errors and universal quantification.\\



\section{Method}
Truth Value Judgment (TVJ) task (\cite{crain2000investigations})
Use both extra object and extra subject picture.\\
Place them in different places of the screen, not always in the bottom right corner.\\
Three word recall task (\cite{cullum1993three}) adult subject are asked to remember three words during the
execution of the experiment to keep their memory occupied.


\subsection{Participants}
English native speakers children between 4 and 5 years old, because they can provide both SR and LG responses.
A group of English native speakers adults of age between 20 and 30 yo (they should have completely developed their linguistic
and cognitive skills). Both groups, children and adults, will be equally divided between male and female subjects.

\subsection{Procedure}
\subsubsection{Children}
Children will be presented a picture in the top centre of a display followed after 2500ms by a spoken sentence,
in the exact moment that they start hearing the sentence they have to move the mouse toward the answer they think it
is correct, in our case the possible answers are "YES/NO" and are presented in two boxes on the left and on the
right of the picture. If the children are too slow in the beginning they will be asked to start moving the mouse earlier.
In order to display the picture and start the single step of the task children will have to click on a start button,
the picture will be presented and then in the same moment when the sentence starts the two boxes containing "YES" and "NO"
will appear on the screen.
Knowing that this can be demanding for the children we take in account the possibility for them to take a break during
the execution of the experiment (approximately in the middle).

\subsubsection{Adults}
% Explain a little bit better
% Verranno date inizialmente 3 parole da ricordare e coincidentemente alla metà
% del test verranno aggiunte altre 3 parole (per un totale di 6) a quelle che devono ricordare.
For adults the TVJ task will be presented in the same way, with the only difference that they won't be able to take
a break during the experiment. Furthermore they will be asked to go through two Three Words Recall (TWR) tasks, one in the
first half and one in the second half of the TVJ task, to keep a portion of their working memory busy and to push them
to give more childlike answers.
If the commitment to universal quantification in children is just bound to their
under fully developed cognitive skills it should be observed in adults a tendency
to give more childlike responses when their working memory is limited.


\subsection{Materials}
\subsubsection{Sentences}
As this study is inspired by the work of \cite{minai2012hinders} the stimuli used for the \textit{TVJ} task are the
english translations of the ones presented in their paper, with the only difference that is necessary to add a
\textit{filler item} in order to have an even number of stimuli, which are necessary to divide the task in two blocks.
The stimuli sentences are therefore 18 and are divided in three groups respectively: 8 target sentences, 8 filler sentences
and 2 warm-up sentences. Below we report an example of a target sentence (\ref{sample_target_sentence}), in (Appendix I) it's possible
to find the complete set of stimuli sentences. All the sentences will be spoken by both a male and a female english native speaker and
recorded in order to be presented to the participants in form of an audio track.

\begin{exe}
    \ex  Every turtle is holding an umbrella. \label{sample_target_sentence}
\end{exe}

Differently from \cite{minai2012hinders} we associated to the target sentences both the under and over-exhaustive scenes. This means
that we have 2 presentable stimuli for each sentence. Participants are presented in the same proportion under and over-exhaustive ones,
respectively four for each type, but they are never presented the same sentence with the two different images, in order to avoid priming.



For this experiment we will use materials very similar to the one used in \cite{minai2012hinders},
with the difference that in addition to under exhaustive scenes, in which objects are more than subjects,
we will provide also over exhaustive ones (i.e. images in which subjects are greater in number than objects).
Another important change to the original images will be done and it will consist in moving
the extra objects, or the extra subjects, respectively, in all the four lattices in which
the image is divided. This hasn't been done in the work of \cite{minai2012hinders} and it could
have led to a lack of processing of the image presented.

We will report some examples of the materials that will be used.
% TODO report examples
\section{Results}
\blindtext

\section{Discussion}
\blindtext


\bibliographystyle{plainnat}

\setlength{\bibhang}{.125in}
\setlength{\bibindent}{-\bibhang}

\vfill
\pagebreak

\bibliography{giacomelli_samuel_pp}

\appendix
\section{Appendix I} \label{appendix_I_sentences}


\end{document}
